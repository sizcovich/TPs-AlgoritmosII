\documentclass[10pt, a4paper]{article}
\usepackage[paper=a4paper, left=1.5cm, right=1.5cm, bottom=1.5cm, top=3.5cm]{geometry}
\usepackage[latin1]{inputenc}
\usepackage[T1]{fontenc}
\usepackage[spanish]{babel}
\usepackage{indentfirst}
\usepackage{fancyhdr}
\usepackage{latexsym}
\usepackage{lastpage}
\usepackage{libs/caratula}
\usepackage{libs/aed2-symb,libs/aed2-itef,libs/aed2-tad}
\usepackage[colorlinks=true, linkcolor=blue]{hyperref}
\usepackage{calc}

\newcommand{\f}[1]{\text{#1}}
\renewcommand{\paratodo}[2]{$\forall~#2$: #1}

\sloppy

\hypersetup{%
 % Para que el PDF se abra a página completa.
 pdfstartview= {FitH \hypercalcbp{\paperheight-\topmargin-1in-\headheight}},
 pdfauthor={C\'atedra de Algoritmos y Estructuras de Datos II - DC - UBA},
 pdfkeywords={TADs b\'asicos},
 pdfsubject={Tipos abstractos de datos básicos}
}

\parskip=5pt % 10pt es el tamaño de fuente

% Pongo en 0 la distancia extra entre ítemes.
\let\olditemize\itemize
\def\itemize{\olditemize\itemsep=0pt}

% Acomodo fancyhdr.
\pagestyle{fancy}
\thispagestyle{fancy}
\addtolength{\headheight}{1pt}
\lhead{Algoritmos y Estructuras de Datos II}
\rhead{$2^{\mathrm{do}}$ cuatrimestre de 2012}
\cfoot{\thepage /\pageref{LastPage}}
\renewcommand{\footrulewidth}{0.4pt}

\author{Algoritmos y Estructuras de Datos II, DC, UBA.}
\date{}
\title{Tipos abstractos de datos b\'asicos}

\begin{document}

\materia{Algoritmos y Estructuras de Datos II}
\submateria{Tipos abstractos de datos b\'asicos}
\titulo{LinkLinkIt}
\subtitulo{Catalogo de rutas en internet}
\grupo{3}
\integrante{Barabas, Ariel}{775/11}{ariel.baras@gmail.com}
\integrante{Izcovich, Sabrina}{550/11}{sizcovich@gmail.com}
\integrante{Otero, Fernando}{424/11}{fergabot@gmail.com}
\integrante{Vita, Sebasti\'an}{149/11}{sebastian\_vita@yahoo.com.ar}

%Pagina de titulo e indice
\thispagestyle{empty}

\maketitle
\tableofcontents

\newpage

%TADS
\section{TAD \tadNombre{ARBOL DE CATEGORIAS}}

\begin{tad}{\tadNombre{AbCat}}
\tadGeneros{AbCat}
\tadExporta{abcat, generadores, observadores básicos}
\tadUsa{\tadNombre{Nombre}}

\tadIgualdadObservacional{a_{1}}{ a_{2}}{arbolcategorias}{$(a_{1} =_{obs} a_{2}
\Leftrightarrow categoria(a_{1}) = categoria(a_{2}) \land subcategorias(a_{1}) =
subcategorias(a_{2}))$}
\tadAlinearFunciones{\argumento = 0?}{nat/$n$,nat/$m$}
\tadObservadores

\tadOperacion{subcategorias}{abcat(nombre)}{conjunto(abcat(nombre))}{}
\tadOperacion{categoria}{abcat(nombre)}{nombre}{}


\tadGeneradores

\tadOperacion{nodo}{nombre x conjunto(abcat)}{abcat}{nombre unico}


\tadOtrasOperaciones


\tadAxiomas
\tadAlinearAxiomas{pred(suc($n$))}
\tadAxioma{categoria(nodo(n, conj(a)))}{n}
\tadAxioma{subcategorias (nodo(n, conj(a)))}{conj(a)}

\end{tad}


\section{TAD \tadNombre{SISTEMA}}

\begin{tad}{\tadNombre{Sist}}
\tadGeneros{Sist}
\tadExporta{sist, generadores, observadores básicos}
\tadUsa{\tadNombre{Fecha, Link, Nombre, Arbol de categorias}}

\tadIgualdadObservacional{ s_{1}}{s_{2}}{sistema}{$(s_{1} =_{obs} s_{2}
\Leftrightarrow (todosLosLinks(s_{1}) = todosLosLinks(s_{2}) \land
 ((\forall: link \in todosLosLinks(s_{1})) visitas(l, s_{1}) = visitas(l, s_{2}) ) \land
((\forall l: link \in todosLosLinks(s_{1})) categoria(l, s_{1}) = categoria(l, s_{2})))$}
\tadAlinearFunciones{\argumento = 0?}{nat/$n$,nat/$m$}
\tadObservadores

\tadOperacion{todosLosLinks}{sist}{conjunto(link)}{}
\tadOperacion{visitas}{link x sist}{conjunto(fecha)}{}
\tadOperacion{categoria}{link x sist}{nombre}{}


\tadGeneradores

\tadOperacion{inicio}{abcat}{sist}{}
\tadOperacion{registrarLink}{link x nombre x sist}{sist}{$link unico \land
nombre en sist$}
\tadOperacion{registrarAcceso}{fecha x link x sist}{sist}{link en sist}


\tadOtrasOperaciones


\tadAxiomas
\tadAlinearAxiomas{pred(suc($n$))}


\end{tad}

\end{document}